\documentclass[a4paper,11pt]{report}

\usepackage[french]{babel}
\usepackage[utf8]{inputenc}
\usepackage[left=2.5cm,top=2cm,right=2.5cm,nohead,nofoot]{geometry}
\usepackage{url}
\usepackage{graphicx}
\usepackage{hyperref}
\usepackage{listings}
\usepackage{amsmath}
\usepackage{amssymb}
\usepackage{color}
\usepackage{listings}
\DeclareMathOperator*{\join}{\bowtie}
\DeclareMathOperator*{\alpham}{\alpha}



\linespread{1.1}

\author{Bruno Rocha Pereira, Antoine Carpentier}
\title{Base de données : Villos}

\begin{document}

\lstset{
    breakatwhitespace=false,
    breaklines=true,
    escapeinside={\%*}{*},
    xleftmargin=5pt,
    xrightmargin=5pt,
    aboveskip=\bigskipamount,
    belowskip=\bigskipamount
}

\maketitle

\chapter{Introduction}

Le but de ce projet est de réaliser une application de gestion des Villos à Bruxelles en utilisant une base de données SQL et de traduire certaines requêtes en SQL, en algèbre relationnelle et en calcul relationnel tuple.
L'application doit permettre à différents types d'utilisateurs de s'inscrire, de se connecter et d'utiliser les Villos et à un administrateur de gérer les paiements qu'ils effectuent, d'effectuer des statistiques et de maintenir l'application.
Un programme permet d'importer les données fournies sous forme de fichiers CSV et XML dans une base de données SQL.
Le premier chapitre détaille les choix effectués quant au langage de programmation, à la base de données et les modifications apportées au modèle de données fourni.
Le deuxième chapitre concerne les requêtes demandées en SQL, en algèbre relationnelle et en calcul relationnel tuple.

\chapter{Choix de conception}

\section{Langages}

Nous avons choisi python car il contient Sqlite3 built-in. Sqlite3 est une bibliothèque qui permet de gérer des bases de données sans avoir besoin de serveurs, en se basant simplement sur des fichiers.
Nous avons également décidé de choisir Flask. Ce framework pour python, associé au moteur de template Jinja, permet de générer facilement et dynamiquement des pages html. Pour complêter les pages html générées, nous avons rajouté du CSS pour les embellir et du Javascript pour pouvoir y ajouter une Google map.

\section{Base de données}

\section{Modèle et importation des données}

\chapter{Requêtes}

\section{Requ\^ete 1}
    \begin{lstlisting}[language=sql]
    SELECT DISTINCT subs.userID, subs.lastname, subs.firstname, subs.addresscity, stations.name FROM trips 
    INNER JOIN subs ON subs.userID = trips.userID
    INNER JOIN stations ON trips.startStation = stations.stationID
    WHERE subs.addresscity='Ixelles' and stations.name = 'FLAGEY';
    \end{lstlisting}

    \begin{align}
    joinedTables \leftarrow trips \bowtie subs \bowtie_{startStation = stationID} \\
    filteredTables \leftarrow \sigma_{addresscity=Ixelles \bigwedge name=FLAGEY}(joinedTables)
    \end{align}

    \begin{align}
    \{ u.userID | subs(u) \wedge \exists t, trips(t) \wedge t.userID=u.userID \exists s, stations(s) \\
     \wedge t.startStation=s \wedge s.name=Flagey \wedge u.addresscity=Ixelles \}
    \end{align}

\section{Requ\^ete 2}
    \begin{lstlisting}[language=sql]
    SELECT DISTINCT trips2.userID
    FROM trips AS trips1
    INNER JOIN trips AS trips2 ON trips1.userID = trips2.userID AND trips1.startTime != trips2.startTime
    ORDER BY trips2.userID;

    \end{lstlisting}

    \begin{align}
    trips1 \leftarrow \alpha_{attributes:t1.attributes}(trips)\\
    trips2 \leftarrow \alpha_{attributes:t2.attributes}(trips)\\
    joinedTables \leftarrow trips1 \bowtie_{t1.userID=t2.userID \bigwedge t1.startTime != t2.startTime} trips2\\
    result \leftarrow \pi{t2.userID}
    \end{align}

    \begin{align}
    \{ u.userID | users(u) \wedge \exists t1, t2, trips(t1), trips(t2) \\
    \wedge t1.userID = t2.userID \wedge t1 != t2 \}
    \end{align}


\section{Requ\^ete 3}
    \begin{lstlisting}[language=sql]
    SELECT DISTINCT trip1.userID, trip2.userID
    FROM trips AS trip1
    INNER JOIN trips AS trip2 ON trip1.startStation = trip2.startStation AND trip1.endingStation = trip2.endingStation AND trip1.userID != trip2.userID;
    \end{lstlisting}

    \begin{align}
    trips1 \leftarrow \alpha_{attributes:t1.attributes}(trips)\\
    trips2 \leftarrow \alpha_{attributes:t2.attributes}(trips)\\
    joinedTables \leftarrow trips1 \bowtie_{t1.startStation = t2.startStation
    \bigwedge t1.endingStation = t2.endingStation \bigwedge t1.userID != t2.userID } trips2 \\
    result \leftarrow \pi_{t1.userID,t2.userID}(joinedTables)
    \end{align}

    \begin{align}
    \{u1.userID, u2.userID | users(u1), users(u2) \\
    \wedge \exists t1, t2, trips(t1), trips(t2) \\
    \wedge t1.userID=u1.userID \wedge t2.userID=u2.userID \\
    \wedge t1.startStation = t2.startStation \\
    \wedge t1.endingStation = t2.endingStation\}
    \end{align}


\section{Requ\^ete 4}
    \begin{lstlisting}[language=sql]
    SELECT DISTINCT trip1.endingStation, trip1.endingTime, trip2.startStation, trip2.startTime
    FROM trips AS trip1
    INNER JOIN trips AS trip2 ON trip1.bicycleID = trip2.bicycleID AND trip1.startTime < trip2.startTime
    GROUP BY trip2.startTime
    HAVING trip1.endingStation != trip2.startStation
    ORDER BY trip1.startTime ASC
    \end{lstlisting}
    \begin{align}
    trips1 \leftarrow \alpha_{attributes:t1.attributes}(trips)\\
    trips2 \leftarrow \alpha_{attributes:t2.attributes}(trips)\\
    joinedTables \leftarrow trips1 \bowtie_{t1.bicycleID = t2.bicycleID \bigwedge t1.startTime < t2.startTime} trips2 \\
    filteredTables \leftarrow \sigma_{t1.endingStation != t2.startStation} (\gamma_{t2.startTime}(joinedTables))\\
    result \leftarrow \pi t1.endingStation, t1.endingTime, t2.startStation, t2.startTime
    \end{align}

    \begin{align}
    \{ b.bicycleID | bicycles(b) \wedge \exists t1, t2, trips(t1), trips(t2) \\
    \wedge t1.bicycleID = b.bicycleID \wedge t2.bicycleID = b.bicycleID \\
    \wedge t2.startTime > t1.startTime \wedge \nexists t3, trips(t3) \\
    \wedge t3.startTime > t1.endingTime \wedge t3.endingTime < t2.startTime \\
    \wedge t1.endingStation != t2.startStation \}
    \end{align}


\section{Requ\^ete 5}
    \begin{lstlisting}[language=sql]
    SELECT users.userID, subs.lastname, subs.firstname,
        (CASE WHEN subs.subscribeDate IS NOT NULL THEN subs.subscribeDate ELSE tempUsers.paymentDate END) AS subscribeDate,
        COUNT(*) AS total_trips,
        SUM(sqrt(power((s1.coordX-s2.coordX)*71, 2) + power((s1.coordY-s2.coordY)*111, 2))) AS total_distance
    FROM trips
    INNER JOIN users ON users.userID = trips.userID
    INNER JOIN stations AS s1 ON s1.stationID = trips.startStation
    INNER JOIN stations AS s2 ON s2.stationID = trips.endingStation
    LEFT OUTER JOIN subs ON subs.userID = users.userID
    LEFT OUTER JOIN tempUsers ON tempUsers.userID = users.userID
    GROUP BY trips.userID
    HAVING total_distance/total_trips
    ORDER BY COUNT(trips.userID);
    \end{lstlisting}


\section{Requ\^ete 6}
    \begin{lstlisting}[language=sql]
    SELECT stations.name, COUNT(trips.endingStation), COUNT(DISTINCT trips.userID)
    FROM stations
    INNER JOIN trips ON trips.endingStation = stations.stationID
    GROUP BY trips.endingStation
    HAVING COUNT(trips.endingStation) >= 10;
    \end{lstlisting}


\end{document}